% Options for packages loaded elsewhere
% Options for packages loaded elsewhere
\PassOptionsToPackage{unicode}{hyperref}
\PassOptionsToPackage{hyphens}{url}
\PassOptionsToPackage{dvipsnames,svgnames,x11names}{xcolor}
%
\documentclass[
  authoryear,
  preprint]{elsarticle}
\usepackage{xcolor}
\usepackage{amsmath,amssymb}
\setcounter{secnumdepth}{5}
\usepackage{iftex}
\ifPDFTeX
  \usepackage[T1]{fontenc}
  \usepackage[utf8]{inputenc}
  \usepackage{textcomp} % provide euro and other symbols
\else % if luatex or xetex
  \usepackage{unicode-math} % this also loads fontspec
  \defaultfontfeatures{Scale=MatchLowercase}
  \defaultfontfeatures[\rmfamily]{Ligatures=TeX,Scale=1}
\fi
\usepackage{lmodern}
\ifPDFTeX\else
  % xetex/luatex font selection
\fi
% Use upquote if available, for straight quotes in verbatim environments
\IfFileExists{upquote.sty}{\usepackage{upquote}}{}
\IfFileExists{microtype.sty}{% use microtype if available
  \usepackage[]{microtype}
  \UseMicrotypeSet[protrusion]{basicmath} % disable protrusion for tt fonts
}{}
\makeatletter
\@ifundefined{KOMAClassName}{% if non-KOMA class
  \IfFileExists{parskip.sty}{%
    \usepackage{parskip}
  }{% else
    \setlength{\parindent}{0pt}
    \setlength{\parskip}{6pt plus 2pt minus 1pt}}
}{% if KOMA class
  \KOMAoptions{parskip=half}}
\makeatother
% Make \paragraph and \subparagraph free-standing
\makeatletter
\ifx\paragraph\undefined\else
  \let\oldparagraph\paragraph
  \renewcommand{\paragraph}{
    \@ifstar
      \xxxParagraphStar
      \xxxParagraphNoStar
  }
  \newcommand{\xxxParagraphStar}[1]{\oldparagraph*{#1}\mbox{}}
  \newcommand{\xxxParagraphNoStar}[1]{\oldparagraph{#1}\mbox{}}
\fi
\ifx\subparagraph\undefined\else
  \let\oldsubparagraph\subparagraph
  \renewcommand{\subparagraph}{
    \@ifstar
      \xxxSubParagraphStar
      \xxxSubParagraphNoStar
  }
  \newcommand{\xxxSubParagraphStar}[1]{\oldsubparagraph*{#1}\mbox{}}
  \newcommand{\xxxSubParagraphNoStar}[1]{\oldsubparagraph{#1}\mbox{}}
\fi
\makeatother


\usepackage{longtable,booktabs,array}
\usepackage{calc} % for calculating minipage widths
% Correct order of tables after \paragraph or \subparagraph
\usepackage{etoolbox}
\makeatletter
\patchcmd\longtable{\par}{\if@noskipsec\mbox{}\fi\par}{}{}
\makeatother
% Allow footnotes in longtable head/foot
\IfFileExists{footnotehyper.sty}{\usepackage{footnotehyper}}{\usepackage{footnote}}
\makesavenoteenv{longtable}
\usepackage{graphicx}
\makeatletter
\newsavebox\pandoc@box
\newcommand*\pandocbounded[1]{% scales image to fit in text height/width
  \sbox\pandoc@box{#1}%
  \Gscale@div\@tempa{\textheight}{\dimexpr\ht\pandoc@box+\dp\pandoc@box\relax}%
  \Gscale@div\@tempb{\linewidth}{\wd\pandoc@box}%
  \ifdim\@tempb\p@<\@tempa\p@\let\@tempa\@tempb\fi% select the smaller of both
  \ifdim\@tempa\p@<\p@\scalebox{\@tempa}{\usebox\pandoc@box}%
  \else\usebox{\pandoc@box}%
  \fi%
}
% Set default figure placement to htbp
\def\fps@figure{htbp}
\makeatother





\setlength{\emergencystretch}{3em} % prevent overfull lines

\providecommand{\tightlist}{%
  \setlength{\itemsep}{0pt}\setlength{\parskip}{0pt}}



 
\usepackage[]{natbib}
\bibliographystyle{elsarticle-harv}


\makeatletter
\@ifpackageloaded{caption}{}{\usepackage{caption}}
\AtBeginDocument{%
\ifdefined\contentsname
  \renewcommand*\contentsname{Table of contents}
\else
  \newcommand\contentsname{Table of contents}
\fi
\ifdefined\listfigurename
  \renewcommand*\listfigurename{List of Figures}
\else
  \newcommand\listfigurename{List of Figures}
\fi
\ifdefined\listtablename
  \renewcommand*\listtablename{List of Tables}
\else
  \newcommand\listtablename{List of Tables}
\fi
\ifdefined\figurename
  \renewcommand*\figurename{Figure}
\else
  \newcommand\figurename{Figure}
\fi
\ifdefined\tablename
  \renewcommand*\tablename{Table}
\else
  \newcommand\tablename{Table}
\fi
}
\@ifpackageloaded{float}{}{\usepackage{float}}
\floatstyle{ruled}
\@ifundefined{c@chapter}{\newfloat{codelisting}{h}{lop}}{\newfloat{codelisting}{h}{lop}[chapter]}
\floatname{codelisting}{Listing}
\newcommand*\listoflistings{\listof{codelisting}{List of Listings}}
\makeatother
\makeatletter
\makeatother
\makeatletter
\@ifpackageloaded{caption}{}{\usepackage{caption}}
\@ifpackageloaded{subcaption}{}{\usepackage{subcaption}}
\makeatother
\journal{Journal Name}
\usepackage{bookmark}
\IfFileExists{xurl.sty}{\usepackage{xurl}}{} % add URL line breaks if available
\urlstyle{same}
\hypersetup{
  pdftitle={Martian Airfoil Design},
  pdfauthor={Justin},
  pdfkeywords={Martian Aerodynamics, Martian Rotorcraft, Computational
Fluid Dynamics},
  colorlinks=true,
  linkcolor={blue},
  filecolor={Maroon},
  citecolor={Blue},
  urlcolor={Blue},
  pdfcreator={LaTeX via pandoc}}


\setlength{\parindent}{6pt}
\begin{document}

\begin{frontmatter}
\title{Martian Airfoil Design \\\large{A Short Subtitle} }
\author[1]{Justin%
%
}
 \ead{s23439.wang@stu.scie.com.cn} 

\affiliation[1]{organization={Shenzhen College of International
Education},,postcodesep={}}

\cortext[cor1]{Corresponding author}

        
\begin{abstract}
Understanding Mar's climatic and geological history is critical for Mars
exploration. Future missions will bases on the data acquired from
unmanned aerial vehicles which can reach terrains inaccessible to
land-base rovers. However, such scientific aerial vehicles are currently
limited by the efficiency of rotor blades due to Mar's thin atmosphere,
where conventional airfoils suffer from laminar separation leading to a
decrease in lift and increase in drag. Sharp-leading edge airfoils are
proposed as a replacement of conventional airfoil as they trigger a
controlled early separation near its leading edge. This separation
produces a low-pressure laminar separation bubble in a time-averaged
sense and subsequent vortex roll-up. However, there lack systematic
evaluation on the effect of thickness reduction on this mechanism. This
study focus on one of the NASA designed and optimised airfoil,
Roamx-0201. The selected airfoil will be tested under various angle of
attack under different flow condition.
\end{abstract}





\begin{keyword}
    Martian Aerodynamics \sep Martian Rotorcraft \sep 
    Computational Fluid Dynamics
\end{keyword}
\end{frontmatter}
    

\section{Introduction}\label{introduction}

The Mars Helicopter Ingenuity have made history in April 2021, by
becoming the first flying machine to perform a controlled flight on
another planet. Though it was initially designed for only five missions,
this solar-powered rotorcraft executed over 71 flights on Mars in three
years, before becoming damaged on its 72th flight in 2024. The
undeniable success of this rotorcraft demonstrates the capability of
ultilizing unmanned aerial vehicles to conduct scientific exploration
missions on other plants. The next generation of Martian rotorcraft is
therefore proposed\citep[\citet{Withrow2020},
\citet{grip2025chopper}]{Young2021}. The new generation of Mars
Helicopter, currently referenced as \textbf{Chopper},
\citep{grip2025chopper}, represents a significant leap forward in
planetary exploration. It is primarily designed for executing complex
scientific missions, allowing the rotorcraft to access previously
unreachable geological terrains. By doing so, it will provide close-up
landscape imagery and collect high-resolution climatic data from
critical locations. However, the heavy mass of the necessary scientific
instruments currently imposes a strict limit on the vehicle's total
payload and thus restricts the scope of these missions.

\begin{figure}

\centering{

\includegraphics[width=1\linewidth,height=\textheight,keepaspectratio]{Figures/ReMachUnification.png}

}

\caption{\label{fig-ReMachUnification}An overview categorizing the Mach
and Reynolds number regimes within which various aerospace articles
operate \citet{leishman2023introduction}. A chart illustrating the
flight airspeed\ldots{}}

\end{figure}%

Ingenuity was integrated with a conventional airfoil,
CLF5605\citep[\citet{zhang2024aerodynamic},
\citet{koning2024experimental}]{caros2025computational}, which however
had limited performance under Martian atmospheric condition. Because the
Martian atmosphere contains more than 95\% CO2 and maintains a low mean
temperature (approximately \(-61^\circ\text{C}\)), the planet exhibits
flight conditions characterized by a significantly elevated Mach number
(\(M = V/\sqrt{\gamma R T}\)). The significantly lower density of the
Martian atmosphere---roughly 1\% that of Earth's---is the principal
factor responsible for the low Reynolds number
(\(Re = \frac{\rho u L}{\mu}\)) experienced on the planet. The Reynolds
number and Mach number on Mars result in a very different flow condition
to Earth's, as demonstrated in Figure \ref{fig:ReMachUnification}. Mars
flow condition has a lower range of Reynolds number and Mach number in
general(\(10^3 \leq Re\leq 10^4\) and \(0.2\leq Ma \leq 1.0\)). With the
same Reynolds number condition, Mars have a higher Mach number compared
to that of Earth's. Therefore, conventional airfoil on Martian
atmosphere experiences a performance drop due to a mechanism called
laminar separation
bubble\citep[\citet{giacomini2024rotorcraft}]{grip2025chopper}. Laminar
separation occurs when the adverse pressure gradient exceeds the
strength of the boundary layer of a laminar flow at the surface of
airfoils. The flow separates permanently and transitions into turbulent
flow, creating small eddies with large kinetic energy. Turbulent flow
change the pressure distribution along the airfoil resulting in large
increase in drag, harming the aerodynamic performance. However, recent
research\citep[\citet{Caros2023}, \citet{caros2025effects},
\citet{Koning2019Airfoil}, \citet{koning2020optimization},
\citet{koning2018low}, \citet{Koning2024ELISA}]{Munday2015Nonlinear}
have looked into methods of reducing the negative impact of laminar
separation bubble by employing sharp-leading edge unconventional
airfoil. Their research have shown, the sharp-leading edge geometries
are able to trigger a low-pressure laminar bubble separation in a
time-averaged sense leading to a flow re-attachment on the airfoil
downstream, creating a low pressure region at the upper surface of an
airfoil.

Koning et al. \citep{Koning2019Airfoil} catagorized airfoils into five
major types, among which cambered and corrugated airfoils are to our
particular interests as they are able to trigger an early laminar
separation bubble which causes flow to re-attach downstream. Triangular
airfoil\citep[\citet{Caros2023},
\citet{caros2025effects}]{caros2022direct} has also demonstrated similar
improved lift mechanism. Koning has identified two airfoils Roamx-0201
and Roamx-1301 using optimisation algorithms\citep{Koning2024ELISA}
through a typical flow condition on Mars at Re=20,000 and M=0.6. These
designs exhibits high lift-to-drag ratio under Martian atmosphere.
However, it lacks robust explanation and systematic evaluation how this
early-separation and re-attachment mechanism are supported by the
sharp-leading edge of airfoils.

In this study, we will focus on the effect of thickness reduction in
shaping the aerodynamic behaviour of airfoil. We hypothesise that a
sharper leading edge will contribute to a more controlled unsteady
laminar separation bubble in a time-averaged sense, leading to a higher
lift-to-drag ratio of airfoil and a better stability during change in
flow conditions of airfoil. To test this, we will take Roamx-0201 and we
will test it under various angle of attack(\(0\leq\alpha\leq6\) at three
different flow conditions corresponding to different sections of rotor
blades: Re = 3,000, M = 0.15; Re = 10,000, M = 0.3; Re = 20,000, M =
0.6. Simulations are conducted using PyFR\citep{PyFR}, which implements
a class of numerical schemes that apply high-order precision to
unstructured meshes.


\bibliography{bibliography.bib}



\end{document}
